\documentclass[12pt]{article}

% packages
\usepackage{setspace}
\usepackage{array}
\usepackage[margin=0.75in]{geometry}
\usepackage{amsmath,bm}
\usepackage{amssymb}
\usepackage{bbold}
\usepackage{physics}
\usepackage{xcolor}
\usepackage{indentfirst}
\usepackage{enumerate}
\usepackage{mathtools}
\usepackage{fancyhdr}
\usepackage{hyperref}


\pagestyle{fancy}
\fancyhf{}
\rhead{Creative Destruction Lab}
\lhead{Introductions to Projects}
\rfoot{Page \thepage}

\allowdisplaybreaks

\title{CDL Quantum 2021 Cohort Project}

\begin{document}

\maketitle

\thispagestyle{empty}
\section{Introduction}

The 2021 Cohort Project is an online collaboration bringing together founders in a series of open-source challenges.
During the four-week bootcamp, founders will form teams to compete and collaborate on a set of challenges on a general topic or theme
relevant for the Quantum Stream.  Four separate challenges will be issued; one per week.  New teams will be formed by CDL Quantum weekly.

Challenges consist of scientific, computational, and business tasks, devised to be tackled in diverse teams of founders with complementary skill sets.  Each year the theme of the Cohort Project will change.  In 2021 it is {\it quantum advantage}.  
Each week, teams will be asked to explore problems and calculations designed to push the boundaries between classical
and quantum computational advantage.
Some of these problems could conceivably be implemented on present-day and near-future quantum devices and computers.
Well designed Cohort Projects could even help set the bar for quantum devices to become useful from a business perspective.

Begin with some suggested reading:
\begin{enumerate}
\item \href{https://www.nature.com/articles/d41586-019-03213-z}{Hello quantum world! Google publishes landmark quantum supremacy claim}
\item \href{https://www.quantamagazine.org/quantum-supremacy-is-coming-heres-what-you-should-know-20190718/}{Quantum Supremacy Is Coming: Here's What You Should Know}
\item \href{https://physicsworld.com/a/towards-a-quantum-advantage/}{Towards a quantum advantage}
\item \href{https://arxiv.org/abs/1801.00862}{Quantum Computing in the NISQ era and beyond}
\end{enumerate}
%\footnote{\href{https://arxiv.org/abs/1812.09976}{\textcolor{blue}{https://arxiv.org/abs/1812.09976} }}


\section{Quantum advantage on near term devices}

Near term or NISQ quantum computers are the devices that we already have today, or can expect to have in the next few years.
To be considered a computer, a quantum device should be programmable to some degree, 
i.e.~it can be provided with coded instructions to run a task of interest.
Despite being programmable, today's quantum computers are a long way from being {\it fault-tolerant}, meaning that errors can be corrected sufficiently
quickly to keep the computer logic stable for arbitrarily long times.  Fault tolerance is required for many famous applications such as 
Shor's or Grover's algorithms, which are known to be faster than classical computational algorithms for certain applications
(like factoring or searching).

The current generation of NISQ quantum computers with tens or hundreds of qubits are not able to correct errors, and hence are not fault tolerant.
As Preskill notes in Ref. [4] above, ``Quantum computers with 50-100 qubits may 
be able to perform tasks which surpass the capabilities of today's classical digital computers, but noise in quantum gates 
will limit the size of quantum circuits that can be executed reliably."  So, what are these tasks that may surpass today's (and tomorrow's)
best classical computers?  

In this project you will explore problems amenable to today's quantum computers.  The question will be addressed, 
how hard is it to achieve {\it quantum advantage} on a useful problem?
You will explore candidate algorithms designed for the some of the most promising quantum
hardware, including superconducting circuits, trapped ions, and arrays of neutral atoms.


\section{Weekly Challenges}

Each week your team will be presented with a set of problems in the form of scientific tasks and challenges.  In order to complete these, you will {\it fork} this repository to a GitHub account managed by your team.  After completing your work on the project at the end of the week, you will issue a {\it pull request} back to the upstream repository.
CDL will choose one team's work to merge into the main upstream repository, where it will become part of the legacy of the 2021 Boot Camp.  For more information see the
GitHub documentation.
\footnote{
    \href{https://help.github.com/en/github/collaborating-with-issues-and-pull-requests/about-forks}{\textcolor{blue}{https://help.github.com/en/github/collaborating-with-issues-and-pull-requests/about-forks} } }

\begin{enumerate}
\item {\it Week 1:} Simulating Quantum Advantage with Trapped Ions
\item {\it Week 2:} Optimization Problems and Rydberg Atom Arrays
\item {\it Week 3:} Variational Quantum Eigensolvers
\item {\it Week 4:} Natural Language Processing

\end{enumerate}


\newpage

\bibliography{refs}
\bibliographystyle{unsrt}

\end{document}
