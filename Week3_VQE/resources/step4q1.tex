\documentclass{article}
\usepackage{hyperref}
\begin{document}
\noindent
Our problem is to minimise $\sum_n \sigma_n^2/N_n$
(the \emph{objective function}) subject to $\sum_n N_n = N_T$
(the \emph{constraint function}).\footnote{%
To simplify notation, we write $\sigma_n$ in place of $\sigma_{H_n}$.}
We can do this using the method of \href{https://en.wikipedia.org/wiki/Lagrange_multiplier}{Lagrange multipliers}.
First we define the Lagrangian function
\begin{equation}
  \mathcal{L} =
  \underbrace{\sum_n \frac{\sigma_n^2}{N_n}}_{\mathrm{objective\ function}}
  - \lambda \underbrace{\left( -N_T + \sum_n N_n \right)}_{\mathrm{constraint\ function}}.
\end{equation}
We then set the partial derivatives of $\mathcal{L}$ with respect to $N_n$ to be zero for each index choice $n$; i.e.
\begin{equation}
  \frac{\partial \mathcal{L}}{\partial N_n} = - \frac{\sigma_n^2}{N_n^2} + \lambda \equiv 0
  \Longrightarrow N_n = \sigma_n/\sqrt{\lambda}.
\end{equation}
We then satisfy the constraint, which is to say we enforce $\frac{\partial \mathcal{L}}{\partial \lambda} = 0$. Thus we require
\begin{equation}
  \sum_n \frac{\sigma_n}{\sqrt{\lambda}} = N_T
  \Longrightarrow \lambda = \left(\frac{\sigma_T}{N_T}\right)^2,
\end{equation}
where we define $\sigma_T := \sum_n \sigma_n$.
Hence the Lagrangian function $\mathcal{L}$ is extremised when
\begin{equation}
   \frac{N_n}{N_T} = \frac{\sigma_n}{\sigma_T},
\end{equation}
which is to say that our optimal splitting would allocate 
a fraction of measurements to outcome $n$ that is roughly equal to the ratio
$\sigma_n/\sigma_T$. This makes sense: we want to allocate more measurements 
to outcomes that are more uncertain.

It is then easy to check that the choice $N_n = \sigma_n/\sigma_T \times N_T$
satisfies Eq.~(3) from \texttt{Instructions.pdf}. Simply substitute our choice
of $N_n$ into the objective function to find
\begin{equation}
  \sum_n \frac{\sigma_n^2}{N_n}
  = \sum_n \sigma_n \times \frac{\sigma_T}{N_T}
  = \frac{\sigma_T^2}{N_T}.
\end{equation}
In other words, our choice saturates the inequality of Eq.~(2) from \texttt{Instructions.pdf} and hence satisfies Eq.~(3)\footnote{%
Eq.~(3) is probably a mistake. We suspect it should be the same as Eq.~(2) except with the inequality replaced with an equality.}.
\end{document}